% Metódy inžinierskej práce

\documentclass[10pt,twoside,slovak,a4paper]{article}

\usepackage[slovak]{babel}
%\usepackage[T1]{fontenc}
\usepackage[IL2]{fontenc} % lepšia sadzba písmena Ľ než v T1
\usepackage[utf8]{inputenc}
\usepackage{graphicx}
\usepackage{url} % príkaz \url na formátovanie URL
\usepackage{hyperref} % odkazy v texte budú aktívne (pri niektorých triedach dokumentov spôsobuje posun textu)

\usepackage{cite}
%\usepackage{times}

\pagestyle{headings}

\title{Využitie platformy Kahoot! pri výučbe\thanks{Semestrálny projekt v predmete Metódy inžinierskej práce, ak. rok 2022/23, vedenie: Vladimír Mlynarovič}} % meno a priezvisko vyučujúceho na cvičeniach

\author{Branislav Sivanič\\[2pt]
	{\small Slovenská technická univerzita v Bratislave}\\
	{\small Fakulta informatiky a informačných technológií}\\
	{\small \texttt{xsivanic@stuba.sk}}
	}

\date{\small 11. október 2022} % upravte



\begin{document}

\maketitle

\begin{abstract}
Článok na tému „Využitie platformy Kahoot! pri výučbe“ bude rozdelený do viacerých kapitol, v ktorých si popíšeme, čo je to vzdelávacia platforma Kahoot!, stručnú históriu tohto programu, možnosti využitia, spôsob, ako dokáže obohatiť výučbu na školách, jeho základné vlastnosti a opis prostredia pre učiteľa a študenta. Jadrom článku budú možnosti využitia tejto platformy vo vzdelávaní, opis vlastnej skúsenosti s touto platformou počas štúdia na základnej a strednej škole a spôsob, ako tento program implementovať do vyučovania na každej modernej škole na Slovensku. Na záver si vyhodnotíme kladné a záporné stránky tejto platformy a uvedieme naše stanovisko k implementácii tejto platformy do výučby na Slovensku.
\end{abstract}



\section{Úvod}
Predmetom môjho článku na tému Využitie platformy Kahoot! pri výučbe je aplikácia Kahoot!, ktorá má široké spektrum využitia v oblasti vyučovania a edukácie študentov počas ich štúdia. Nosnou časťou článku je princíp fungovania platformy Kahoot!, kde je možné ju využiť a spôsoby obohatenia vyučovania pomocou tejto aplikácie. V kapitole ~\ref{kahoot} je podrobnejšie popísané, čo je to aplikácia Kahoot!, na čo sa používa a história jej vzniku. V ďalšej časti ~\ref{vyuzitie} podrobne opisujem možnosti, kde sa táto platforma dá využiť a taktiež uvediem reálne príklady, kedy sa platforma Kahoot! najviac využívala. Nasledujúca časť ~\ref{obohatenie} sa venuje rôznym možnostiam obohatenia klasickej výučby na školách alebo stretnutiach. Podkapitola ~\ref{obohatenie:priklady} opisuje konkrétne príklady, kedy a ako je možné obohatiť často monotónnu vyúčbu. Kapitola ~\ref{skusenost} sa venuje opisu mojich skúseností s touto výučbovou platformou a vlastného názoru, či mi počas môjho štúdia pomohla. V časti kladné a záporné stránky ~\ref{plusyminusy} sú popísané výhody a nevýhody platformy Kahoot! pri výučbe na školách. Záverečná časť ~\ref{zaver} sa venuje zhrnutiu nadobudnutých poznatkov ohľadom platfotmy Kahoot! a jej využitia.


\section{Platforma Kahoot!} \label{kahoot}

Z obr.~\ref{f:rozhod} je všetko jasné. 

\begin{figure*}[tbh]
\centering
%\includegraphics[scale=1.0]{diagram.pdf}
Aj text môže byť prezentovaný ako obrázok. Stane sa z neho označný plávajúci objekt. Po vytvorení diagramu zrušte znak \texttt{\%} pred príkazom \verb|\includegraphics| označte tento riadok ako komentár (tiež pomocou znaku \texttt{\%}).
\caption{Rozhodujúci argument.}
\label{f:rozhod}
\end{figure*}



\section{Využitie} \label{vyuzitie}

Základným problémom je teda\ldots{} Najprv sa pozrieme na nejaké vysvetlenie (časť~\ref{ina:nejake}), a potom na ešte nejaké (časť~\ref{ina:nejake}).\footnote{Niekedy môžete potrebovať aj poznámku pod čiarou.}

Môže sa zdať, že problém vlastne nejestvuje\cite{Coplien:MPD}, ale bolo dokázané, že to tak nie je~\cite{Czarnecki:Staged, Czarnecki:Progress}. Napriek tomu, aj dnes na webe narazíme na všelijaké pochybné názory\cite{PLP-Framework}. Dôležité veci možno \emph{zdôrazniť kurzívou}.

\section{Obohatenie výučby} \label{obohatenie}

\subsection{Príklady} \label{obohatenie:priklady}

Niekedy treba uviesť zoznam:

\begin{itemize}
\item jedna vec
\item druhá vec
	\begin{itemize}
	\item x
	\item y
	\end{itemize}
\end{itemize}

Ten istý zoznam, len číslovaný:

\begin{enumerate}
\item jedna vec
\item druhá vec
	\begin{enumerate}
	\item x
	\item y
	\end{enumerate}
\end{enumerate}


\subsection{Ešte nejaké vysvetlenie} \label{ina:este}

\paragraph{Veľmi dôležitá poznámka.}
Niekedy je potrebné nadpisom označiť odsek. Text pokračuje hneď za nadpisom.



\section{Vlastná skúsenosť} \label{skusenost}




\section{Kladné a záporné stránky} \label{plusyminusy}




\section{Záver} \label{zaver} % prípadne iný variant názvu



%\acknowledgement{Ak niekomu chcete poďakovať\ldots}


% týmto sa generuje zoznam literatúry z obsahu súboru literatura.bib podľa toho, na čo sa v článku odkazujete
\bibliography{literatura}
\bibliographystyle{plain} % prípadne alpha, abbrv alebo hociktorý iný
\end{document}
